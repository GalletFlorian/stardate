\section{Conclusion}
\label{section:conclusion}

We present a statistical framework for measuring precise ages of MS stars and
subgiants by combining observables that relate, via different evolutionary
processes, to stellar age.
Specifically, we combined HR diagram/CMD placement with rotation periods, in a
hierarchical Bayesian model, to age-date stars based on both their hydrogen
burning and magnetic braking history.
The two methods of isochrone fitting and gyrochronology were combined by
taking the product of two likelihoods: one that contains an isochronal model
and the other a gyrochronal one.
We used the MIST stellar evolution models and computed isochronal ages and
likelihoods using the {\tt isochrones.py} {\it Python} package.
The gyrochronal model was a power-law relation between rotation period, B-V
color and age, based on the functional form first introduced by
\citet{barnes2003} and later recalibrated by \citet{angus2015}, with a
modification recommended by \citet{vansaders2016} that accounts for weakened
magnetic braking at Rossby numbers larger than 2.
We tested this age-dating method, implemented in a {\it Python} package called
\sd, on simulated data and cluster stars.
We demonstrated that combining gyrochronology with isochrone fitting produces
age predictions that are an order of magnitude more precise than isochrone
fitting alone.
Gyrochronology and isochrone fitting are complementary: gyrochronology
supplies precise ages on the \MS\ and isochrone fitting provides precise ages
near \MS\ turn off.
\sd\ allows users to infer precise ages for MS stars and subgiants alike,
without having to first identify the age-dating method that is best for any
given star.
In addition, due to the flexibility of the {\tt isochrones} package that \sd\
is built on top of, \sd\ accepts apparent magnitudes in all pass-bands covered
by the MIST isochrones which includes the Johnson-Cousins, {\it 2MASS},
\Kepler, {\it SDSS} and \Gaia\ photometric systems.
However, we caution users that the gyrochronology model currently built into
\sd\ does not alway provide a good fit to all data and is not suitable for
very young stars or binaries.
In the future we hope to make several improvements to the gyrochronology
relation implemented in \sd\ that will make it applicable to {\it all} MS and
subgiant stars.

The code used in this project is available as a documented {\it python}
package called \sd.
It is available for download via Github\footnote{git clone
https://github.com/RuthAngus/stardate.git} or through
PyPI\footnote{pip install stardate\_code}.
Documentation is available at https://stardate.readthedocs.io/en/latest/.
All code used to produce the figures in this paper is available at
https://github.com/RuthAngus/stardate.
\racomment{add github hash and Zenodo doi}.
