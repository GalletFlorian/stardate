We present a new age-dating technique that combines gyrochronology and
isochrone fitting to infer ages for FGKM main-sequence and subgiant field
stars.
The age-dating methods of gyrochronology and isochrone fitting are each
capable of providing relatively precise ages in certain areas of the
Hertzsprung-Russell diagram: rotation periods can provide precise ages for
cool stars on the main sequence via gyrochronology, and isochrone fitting
can provide precise ages for stars near main sequence turn off.
Combined, these two age-dating techniques can be applied to a broader range of
stellar masses and evolutionary stages and can provide ages that are more
precise and accurate than either method used in isolation.
In this investigation, we demonstrate that the position of a star on the
Hertzsprung-Russell or color-magnitude diagram can be combined with its
rotation period to infer a precise age via both isochrone fitting and
gyrochronology simultaneously.
We show that incorporating rotation periods with 5\% uncertainties into
stellar evolution models particularly improves age precision and accuracy for
FGK stars on the main sequence, and can provide age estimates that are up to
three times more precise than isochrone fitting alone for these stars.
In addition, we provide a new gyrochronology relation, calibrated to the
Praesepe cluster and the Sun, that includes a variance model to capture the
rotational behavior of stars without Sun-like magnetic dynamos.
This publication is accompanied by an open source {\it Python} package (\sd)
for inferring the ages of main sequence and subgiant FGKM stars from rotation
periods, spectroscopic parameters and/or apparent magnitudes and parallaxes.
