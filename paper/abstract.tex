We present a new age-dating technique that combines gyrochronology with
isochrone fitting to infer ages for FGKM main-sequence and subgiant field
stars.
Gyrochronology and isochrone fitting are each capable of providing relatively
    precise ages for field stars in certain areas of the Hertzsprung-Russell
    diagram: gyrochronology works optimally for cool main-sequence stars, and
    isochrone fitting can provide precise ages for stars near the
    main-sequence turnoff.
Combined, these two age-dating techniques can provide precise and accurate
ages for a broader range of stellar masses and evolutionary stages than either
method used in isolation.
We demonstrate that the position of a star on the Hertzsprung-Russell or
color-magnitude diagram can be combined with its rotation period to infer a
precise age via both isochrone fitting and gyrochronology simultaneously.
We show that incorporating rotation periods with 5\% uncertainties into
    stellar evolution models improves age precision for FGK stars on the main
    sequence, and can, on average, provide age estimates up to three times
    more precise than isochrone fitting alone.
In addition, we provide a new gyrochronology relation, calibrated to the
Praesepe cluster and the Sun, that includes a variance model to capture the
rotational behavior of stars whose rotation periods do not lengthen with the
square-root of time, and parts of the Hertzsprung-Russell diagram where
gyrochronology has not been calibrated.
This publication is accompanied by an open source {\it Python} package (\sd)
for inferring the ages of main-sequence and subgiant FGKM stars from rotation
periods, spectroscopic parameters and/or apparent magnitudes and parallaxes.
