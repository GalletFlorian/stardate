We present a new method for inferring stellar ages that combines two different
established age-dating methods: gyrochronology and isochrone fitting, and
provides ages with 20\% uncertainties across the MS and subgiant branch.
Gyrochronology and isochrone fitting are independent age-dating methods, each
capable of providing extremely precise ages in certain areas of the
Hertzsprung-Russell diagram.
Combined, they can be applied to a much broader range of stellar masses and
evolutionary stages and can provide ages that are more precise and accurate
than either method in isolation.
Rotation periods supply precise ages for cool stars on the main sequence via
gyrochronology and isochrone fitting provides precise ages near main sequence
turn off.
In this investigation, the observables of main sequence stars that are used to
trace core hydrogen burning and stellar evolution on the Hertzprung-Russell
diagram (\teff, \feh, \logg, parallax, apparent magnitude and photometric
colors) are combined with \kepler\ rotation periods, in a Bayesian framework,
to jointly infer stellar ages from both isochrone fitting and gyrochronology
simultaneously.
We show that incorporating rotation periods into stellar evolution models
significantly improves the precision of inferred ages on the main sequence.
However, since ages predicted with gyrochronology on the main sequence are, in
general, much more precise than isochronal ages, care must be taken to ensure
the gyrochronology relation being used is accurate.
The goal of this study was to explore the process of combining two independent
dating methods, not to recalibrate or improve upon existing gyrochronology
models.
However, only a slight modification to our algorithm would be required to
perform a calibration and, since the code is modular, an updated
gyrochronology model could easily replace the one used here in future.
This publication is accompanied by open source code ({\it Python} package,
\sd), for inferring stellar ages for cool main sequence stars and subgiants
from rotation periods, spectroscopic parameters and/or apparent magnitudes and
parallaxes.
