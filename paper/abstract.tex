We present a new age-dating technique that combines gyrochronology and
isochrone fitting to infer ages for FGKM main-sequence and subgiant field
stars.
Gyrochronology and isochrone fitting are two age-dating methods that are each
capable of providing relatively precise ages in certain areas of the
Hertzsprung-Russell diagram.
Rotation periods can provide precise ages for cool stars on the main sequence
via gyrochronology, and isochrone fitting provides precise ages for stars near
main sequence turn off.
Combined, these two age-dating techniques can be applied to a broader range of
stellar masses and evolutionary stages and can provide ages that are more
precise and accurate than either method used in isolation.
In this investigation, we demonstrate that the position of a star on the
Hertzsprung-Russell or color-magnitude diagram can be combined with its
\kepler\ rotation period to infer a precise age from both isochrone fitting
and gyrochronology simultaneously.
We show that incorporating rotation periods into stellar evolution models
particularly improves age precision and accuracy for GK stars on the main
sequence, and can result in age estimates that are three times more precise
and accurate than isochrone fitting alone.
In addition, we provide a new gyrochronology relation, calibrated to the
Praesepe cluster and the Sun, that includes a variance model for stars that
do not have Solar-like magnetic dynamos.
This publication is accompanied by an open source {\it Python} package (\sd)
for inferring the ages of main sequence and subgiant FGKM stars from rotation
periods, spectroscopic parameters and/or apparent magnitudes and parallaxes.
