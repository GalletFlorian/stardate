\section{Introduction}
\label{section:intro}

Age is the most difficult stellar property to measure, and the difficulty of
age-dating is particularly acute for low mass (GKM) stars on the main sequence
(MS).
Using conventional dating methods, uncertainties on the ages of these stars
can be as large as the age of the Universe.
GKM dwarfs are difficult to age-date because their physical and observable
properties do not change rapidly.
This is represented in the spacing of isochrones on a Hertzsprung-Russell
diagram (HRD) or color-magnitude diagram (CMD).
On the MS, isochrones are tightly spaced and, even with very precise
measurements of effective temperature and luminosity, the position of a MS
star on the HRD may be consistent with range of isochrones spanning several
billion years.
At main sequence turn-off however, isochrones are spread further apart, so
that sufficiently precisely measured temperatures and luminosities can yield
ages that are extremely precise \citep[\eg][]{pont2004}.
% Typical age uncertainties of dwarfs.
The classical method for measuring stellar ages is isochrone placement, or
isochrone fitting, where surface gravity changes resulting from fusion in the
core (usually observed via luminosity, $L$, and effective temperature, \teff,
or absolute magnitude and colour) are compared with a set of models that trace
stellar evolution across the HRD, or CMD.
CMD/HRD position has been thoroughly mapped with physical models, and can be
used to calculate relatively accurate (but not necessarily precise) ages,
barring some small, $\sim$10\% variations between different models,
\citep[\eg][]{yi2001, dotter2008, dotter2016}.
On the MS itself, there is little differentiation between stars of different
ages in the $L$ and \teff\ plane, so ages tend to be very imprecise.
The method of inferring a star's age from its rotation period, called
`gyrochronology', is much better suited for measuring ages on the MS because
MS stars spin down relatively rapidly.

Magnetic braking in MS stars was first observed by \citet{skumanich1972} who,
studying young clusters and the Sun, found that the rotation periods of
Solar-type stars decay with the square-root of time.
It has since been established that the rotation period of a MS star depends,
to first order, only on its age and effective temperature or color
\citep[\eg][]{barnes2003}.
The convenient characteristic of stars that allows their ages to be inferred
from their {\it current} rotation periods and independently of their
primordial ones, comes from the steep dependence of spin-down rate on rotation
period \citep{kawaler1989}.
Stars spinning with high angular velocity will experience a much greater
angular momentum loss rate than slowly spinning stars and for this reason, no
matter the initial rotation period, Solar type stars will have the same
rotation period after around the age of the Hyades, 500-700 million years
\citep{irwin2009, gallet2015}.
After this time, the age of a star can be inferred, to first order, from its
dust-corrected color (\eg\ B-V or \gcolor) and rotation period alone and this
is the principle behind gyrochronology.

The relation between age, rotation period and mass has been studied in detail,
and several different models have been developed to capture the rotational
evolution of Sun-like stars.
Some of these models are theoretical and based on physical processes; modeling
angular momentum loss as a function of stellar properties as well as the
properties of the magnetic field and stellar wind \citep[\eg][]{kawaler1988,
kawaler1989, vansaders2013, matt2015, vansaders2016}.
Other models are empirical and capture the behavior of stars from a purely
observational standpoint, using simple functional forms that can reproduce the
data \citep[\eg][]{barnes2003, barnes2007, mamajek2008, angus2015}.
Both types of model, theoretical and empirical, must be calibrated using
observations.
Old calibrators are especially important because new evidence suggests that
rotational evolution goes through a transition at old age or, more
specifically, at a large Rossby number, $Ro$ (the ratio of rotation period to
the convective overturn timescale).
For example, stars shown to be old from \kepler\ asteroseismic data rotate
more rapidly than expected given their age \citep{angus2015, vansaders2016}.
A new physically motivated gyrochronology model, capable of reproducing these
data, was recently introduced \citep{vansaders2016}.
It relaxes magnetic breaking at a critical Rossby number of around 2,
approximately the Solar value.
This model predicts that, after stellar rotation periods lengthen enough to
move stars across this $Ro$ threshold, stars stop spinning down and maintain a
constant rotation period from then until they evolve off the MS.
As demonstrated in section \ref{section:results} however, this does not mean
that the ages of stars with Ro $>$ 2 cannot be measured from their rotation
periods, especially if their rotation periods are combined with HRD or CMD
placement.

The gyrochronology models that capture post $Ro$-threshold, rotational
evolution \citep{vansaders2016} are the current state-of-the-art in rotation
dating.
These models can be computed over a grid of stellar parameters, and
interpolated over to predict the age of a star.
The process of measuring the age of a field star with these models is
similar to inferring an age using any set of isochrones, with difference that
rotation period is an additional observable dimension.
Ages calculated using these models are therefore likely to be much more
precise than using rotation-free isochrones since rotation period provides an
additional anchor-point for the age of a star.
We present here a complementary method that combines isochrones with an {\it
empirical} gyrochronology model.
The methodology is related to the \citet{vansaders2016} model in that both use
a combination of rotation periods and other observable properties that track
stellar evolution on the HRD in concert.
The main difference is that the gyrochronology model used here is an entirely
empirically calibrated one, as opposed to a physically derived one.
One major advantage of using a physically motivated gyrochronology model is
the ability to rely on physics to interpolate or extrapolate over parts of
parameter space with sparse data coverage.
However, rotational spin-down is a complex process that is not yet fully
understood and currently no physical model can accurately reproduce all the
data available.
For this reason, even physically motivated gyrochronology models cannot always
be used to reliably extrapolate into unexplored parameter space.
Physical models, when calibrated to data, can provide insight into the physics
of stars, however, if accurate and precise {\it prediction} of stellar
properties is desired, empirical models can have advantages over physical
ones.
For example, the data may reveal complex trends that cannot be reproduced with
our current understanding of the physical processes involved, but may be
captured by more flexible data-driven models.
In addition, it is relatively straightforward to build an element of
stochasticity into empirical models, \ie\ to allow for and incorporate
outliers or noisy trends.
This is particularly important for stellar spin down because rotation periods
can be affected by additional confounding variables which are not always
observed (having a binary or planetary companion, for example).
A further advantage of empirical models is that inference is more tractable:
it can be extremely fast to fit them to data.

In this work we calibrated a new empirical gyrochronology relation, fit to the
Praesepe open cluster and the Sun, in \gaia\ \gcolor\ color.
\Gaia\ $G$, $G_{BP}$ and $G_{RP}$ apparent magnitudes are now the most
abundant photometric measurements, available for more than a billion stars
\citep{brown2018}.
\racomment{In fact, an important point of context for this work is the new
availability of data relevant to gyrochronology and stellar ages.
\Gaia\ now provides broad-band photometry and parallaxes for over a billion
stars, and \kepler, \ktwo\ and \tess\ are providing rotation periods for
hundreds of thousands of stars.
Gyrochronology is becoming one of the most readily available age-dating
methods, so continuing to improve gyrochronology relations and methods is
important.}
Like any other gyrochronology relation, this new Praesepe-based model does not
perfectly reproduce all the observed data and some simple modifications could
make significant improvements, for example, by including a mixture model to
account for outliers and binaries, and by removing the period-age,
period-color separability to account for different period-color shapes seen in
clusters of different ages.
We leave these improvements for a future project and, for now, test this
new Praesepe-based model, which is built into an open source {\it Python}
package for gyrochronology called \sd.
\sd\ provides the framework for simultaneous gyrochronology and isochrone
fitting, and, because \sd\ is modular, it would be straightforward to update
the gyrochronology relation in future.

This paper is laid out as follows.
In section \ref{section:method} we describe our new age-dating model and its
implementation, in section \ref{section:results} we test this model on
simulated stars and cluster stars, and in section \ref{section:conclusion} we
discuss the implications of these tests and future pathways for development.
Throughout this paper we use the term {\it `observables'} to refer to the
following observed properties of a star, \teff, \logg, observed bulk
metallicity ($[\hat{\mathrm{Fe/H}}]$), parallax ($\bar{\omega}$), photometric
colors in different passbands (${\bf m_x} = [m_J, m_H, m_K, m_B, m_V, m_G,
m_{GBP}, m_{GRP}...]$, etc) and rotation period ($P_{\mathrm{rot}}$).
The term `parameters' refers to the physical properties of that star: age
($A$), equivalent evolutionary phase (EEP), metallicity ($\mathrm{Fe/H}$),
distance ($D$) and V-band extinction ($A_V$).
These are the properties that generate the observables.
