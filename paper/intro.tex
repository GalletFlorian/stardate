\section{Introduction}
\label{section:intro}

% Motivation and context
%-----------------------

%   - The need for better stellar ages.
The formation and evolution of the Milky Way (MW) and the planetary systems
within it are two topics of significant interest in astronomy today.
Both of these fields require precise and accurate ages for tens to hundreds of
thousands of stars.
% Precise ages of red giant stars, some calculated from asteroseismology, some
% from spectroscopy, some from physical models and some from machine learning
% methods have recently been used to explore the age distributions of stellar
% populations in the MW \citep[\eg][]{ness2015, stello2017, das2018,
% sanders2018}.
% Red giants are highly luminous and can be observed to great distances, thus
% providing age information on the scale of tens of kilo-parsecs.
% Main sequence (MS) stars on the other hand, although fainter, are more
% numerous and so their ages may provide new insights into the formation and
% evolution of the Solar neighborhood.
% MS star ages are also of great interest for studying the formation and
% evolution of planetary systems.
% Almost all exoplanets discovered to date orbit MS stars and it is therefore
% {\it MS star} ages that are needed to study planet evolution.
% Unfortunately, the very property that makes MS stars good hosts for habitable
% planets also makes them difficult to date: they do not change substantially
% over time.
% There are several properties of giant stars that make them easier to date than
% MS stars.
% One property is that their acoustic oscillations have long periods of hours to
% days and large amplitudes, so they can be detected and resolved using the
% \kepler\ spacecraft's long cadence mode.
% The acoustic modes of MS stars, on the other hand, have typical periods of
% five minutes, which are unresolvable in \kepler's long cadence mode (thirty
% minutes between observations).
% The large numbers of giants with asteroseismic ages \citep[\eg][]{huber2010,
% yu2016, stello2017, yu2018, de_assis_peralta2018} provides a large training
% set for machine learning methods.
% In addition the ages of red giants are strongly correlated with their masses,
% so if mass can be measured (from the spectrum, for example), so can age
% \citep{ness2015}.
% Age is the most elusive fundamental stellar property and is extremely
% difficult to measure for field stars -- especially cool stars on the MS.
% However, a time axis is essential for studying the evolution and formation of
% any astronomical object, from a planetary system to the Milky Way itself, so
% improvements in stellar age-dating methods would have a wide-reaching impact
% across astronomy.
% For example, in the field of exoplanets the ages of planet hosting stars could
% be used to explore the dynamical and architectural evolution of planetary
% systems.
% % The typical timescales for planetary system stability could be inferred,
% % leading to implications for the fates of planetary systems in the Milky Way
% % including our own.
% New stellar ages would also benefit galactic archaeologists and could resolve
% the mystery behind the formation of the galactic thin disk.
% {\bf The need for more precise and accurate stellar ages is virtually
% universal within astronomy.}

The difficulty of age-dating is particularly acute for low mass (GKM) stars on
the MS: precisely those that comprise the majority of known planet hosts.
Using conventional dating methods, uncertainties on the ages of these stars
can be as large as the age of the Universe, in other words they are completely
unconstrained.
% , and
% ages precise to within 50\% are considered `ultra-precise'
% \citep{soderblom2010}.
The stars eligible for {\it truly} precise age-dating, where age uncertainties
can be as low as 10\%, are those in nearby open clusters, those with
observable acoustic oscillations (asteroseismic stars), those just turning off
the MS, and the Sun
\citep[see][for a review of stellar ages]{soderblom2010}.
There are only a few tens of cool, MS stars with precise ages that are
suitable for exoplanet population studies, however {\it tens-of-thousands} of
precise ages are needed to study the evolution of planetary systems
\citep[\eg][]{petigura2013, foreman-mackey2014, veras2015, burke2015}.
The number of planets detected in open clusters, therefore with precise ages,
is growing, however there are still only a couple of dozen of these discovered
so far and the total number of detectable planets in clusters is unlikely to
reach statistical numbers in the near future, if ever.
In order to study the evolution of planetary systems, a significant number of
precise ages for cool MS {\it field} stars are needed.
Neither cluster stars nor asteroseismic stars can currently provide the
numbers required for exoplanet population studies: age-dating methods for cool
MS field stars {\it must} be improved before the evolution of planets can be
explored.

% Planetary system studies are one example of an astronomical field that would
% greatly benefit from a large stellar age catalog.
% Another notable field that requires large numbers of stellar ages is galactic
% archaeology.
% Machine learning methods have significantly improved age-dating for field red
% giant stars recently, and large-scale catalogs of precise giant ages have lead
% to dramatic breakthroughs in galactic archaeology \citep[\eg][]{ness2015,
% das2018, sanders2018}, but these methods are not yet applicable to MS stars,
% yet MS stars are more numerous than red giants and their ages could
% revolutionize galactic archaeology.
% The largest catalog of precise ages for field {\it MS} stars currently
% available is the short-cadence Kepler asteroseismic catalog of 500 stars
% \citep{chaplin2014}.
% However, few of these stars are truly on the MS (most are on the subgiant
% branch) and even fewer are cool -- most being A, F and G types.
% There is currently {\it no} large catalog of stellar ages for field MS stars,
% and there is unlikely to be one until after the launch of {\it PLATO} and the
% release of a large catalog of asteroseismic properties.

%   - Why are MS ages harder than red giant ages?
% Unlike red giants, the masses of MS stars are not strongly correlated with
% age, so mass cannot be used as an age proxy.
The spectra and colors of MS stars do not contain a significant amount of age
information because they do not change rapidly.
This is represented in the spacing of isochrones on a Hertsprung-Russell (HR)
or color-magnitude diagram (CMD).
On the MS, isochrones are tightly spaced and, even with very precise
measurements of effective temperature and luminosity, the position of a MS
star on the HR diagram may be consistent with range of isochrones spanning
several billion years.
At main sequence turn-off however, isochrones are spread further apart, so
that sufficiently precisely measured temperatures and luminosities may yield
ages that are extremely precise.
% Typical age uncertainties of dwarfs.
The classical method for measuring stellar ages is called isochrone placement,
or isochrone fitting, where surface gravity changes resulting from fusion in
the core (usually observed via luminosity, $L$, and effective temperature,
\teff, or absolute magnitude and colour) are compared with a set of models
that trace stellar evolution across the Hertzprung-Russell, or color-magnitude
diagram (CMD).
Surface gravity changes have been thoroughly mapped with physical models, and
can be used to calculate relatively accurate (but not necessarily precise)
ages, barring some small, $\sim$10\% variations between different models
\citep[\eg][]{yi2001, dotter2008, dotter2016}.
Isochronal ages {\it can} be precise for stars turning off the MS, because the
rate of change in brightness and temperature is large during this phase of
stellar evolution.
However, on the MS itself, there is little differentiation between stars of
different ages in the $L$ and \teff\ plane, so ages tend to be very imprecise.
The method of inferring a star's age from its rotation period, called
`gyrochronology', is much better suited for measuring ages on the MS because
MS, stars spin down relatively rapidly.

%   - How does gyrochronology work?
Magnetic braking in MS stars was first observed by \citet{Skumanich1972} who,
studying young clusters and the Sun, found that the rotation periods of
Solar-type stars decay with the square-root of time.
It has since been established that the rotation period of a star depends, to
first order, only on its age and mass \citep[\eg][]{barnes2003}.
This means that by measuring a star's rotation period and a suitable mass
proxy (B-V color is commonly used), one can determine its age.
The convenient characteristic of stars that allows their ages to be inferred
from their {\it current} rotation periods and independently of their
primordial ones, comes from the steep dependence of spin-down rate on rotation
period \citep{kawaler1989}.
% Observations of young clusters indicate that stellar angular momentum loss
% rate is proportional to the cube of the angular velocity.
This means that a star spinning with high angular velocity will experience a
much greater angular momentum loss rate than a slowly spinning star.
For this reason, no matter the initial rotation period of a Sun-like star,
after around the age of the Hyades (500-700 million years) stellar rotation
periods appear to converge onto a tight sequence \citep{irwin2009}.
After this time, the age of a star can be inferred, to first order, from its
mass and rotation period alone and this is the principle behind gyrochronology.

% % Explain the project (add more detail here.)
% In this work we present and test a new age-dating model that combines stellar
% evolution models with an empirical gyrochronology model in a Bayesian
% framework.
% This approach has an advantage over either isochrone fitting or gyrochronology
% alone since the addition of age-relevant information will always improve
% stellar age estimates, unless the models are wrong.

%   - Theoretical vs empirical gyro models
The relation between age, rotation period and mass has been studied in detail,
and several different models have been developed to capture the rotational
evolution of Sun-like stars.
Some of these models are theoretical and based on physical processes; modeling
angular momentum loss as a function of the stellar properties as well as the
properties of the magnetic field and stellar wind \citep{kawaler1988,
kawaler1989, vansaders2013, matt2015, vansaders2016}.
Other models are empirical and capture the behavior of stars from a purely
observational standpoint, using simple functional forms that can reproduce the
data \citep{barnes2003, barnes2007, mamajek2008, angus2015}.
Both types of model, theoretical and empirical, must be calibrated using
observations.
% Even the theoretical models are highly sensitive to some stellar properties
% that are not measurable: mass-loss rate and magnetic field geometry, for
% example.
Old calibrators are especially important because new evidence suggests that
rotational evolution goes through a transition at old age or, more
specifically, at a large Rossby number, $Ro$ (the ratio of rotation period to
the convective overturn timescale).
For example, old \kepler\ asteroseismic stars rotate more rapidly than
expected given their age \citep[\eg][]{angus2015, vansaders2016}.
A new physically motivated gyrochronology model, capable of reproducing these
data, was recently introduced \citep{vansaders2016}.
It relaxes magnetic breaking at a critical Rossby number of around the Solar
value, 2.1.
This model predicts that, after stellar rotation periods lengthen enough to
move stars cross this $Ro$ threshold, stars stop spinning down and maintain a
constant rotation period from then until they evolve off the MS.
The implication is that the ages of stars with $Ro >$ 2.1 cannot be measured
from their rotation periods.
% Despite recent advances in rotation-dating, such as the new $Ro$ transition
% models \citep{vansaders2016, vansaders2018}, there is still substantial room
% for improvement in the gyrochronology relations.

% %   - Motivation for improvements to gyrochronology relations
% % For example, there is, as yet no {\it fully} empirical gyrochronology model
% % that includes a weakened magnetic braking law after the $Ro$ = 2.1 transition.
% Empirical models have value as they are often more flexible than physical
% models and can capture trends in the data before the physical processes at
% work are fully understood.
% In addition, no single gyrochronology relation can reproduce the rotational
% behaviour of all open clusters; the relation between rotation period, age and
% mass varies from cluster to cluster in a way that cannot be captured by
% any current model.
% Again, this is related to the issue of model flexibility.
% Finally, more stars with precisely measured old ages are needed to confidently
% calibrate gyrochronology models at old ages.
% New asteroseismic calibrators will become available from reprocessed \kepler\
% short cadence light curves, however significant numbers of suitable
% gyrochronology calibrators may not be accessible until after the European
% Space Agency's \plato\ mission is launched.

%   - Describe the project presented here.
The gyrochronology models that capture post $Ro$-threshold, rotational
evolution \citep{vansaders2016} are the current state-of-the-art in rotation
dating.
% developed and calibrated in \citet{epstein2014, vansaders2015,
% vansaders2016, vansaders2018} However, despite their superior accuracy, they
These models are expensive to compute and, just as with most isochrones and
stellar evolution tracks, are usually pre-computed over a grid of stellar
parameters, then interpolation is used to predict the age of a star.
The process of measuring a stellar age with these models is similar to
inferring an age using any set of isochrones, with the main difference being
that rotation period is an additional dimension.
Ages calculated using these models are therefore likely to be much more
precise than using rotation-free isochrones since rotation period provides an
additional anchor-point for the age of a star.
% In order to infer an age from these models, one would effectively perform
% isochrone fitting but in this case, rotation period would be added as an
% additional parameter and the ages inferred would therefore be more precise and
% accurate.
We present here a complementary method that combines isochrones with an {\it
empirical} gyrochronology model using a Bayesian framework.
The methodology is related to the models described above \citep{vansaders2016}
in that both use a combination of rotation periods and other observable
properties that track stellar evolution on the HR diagram in concert.
The main difference is that the gyrochronology model used here is an entirely
empirically calibrated one, as opposed to a physically derived one.
One major advantage of using a physically motivated gyrochronology model over
an empirically calibrated one is the ability to rely on physics to interpolate
or extrapolate over parts of parameter space with sparse data coverage.
However, rotational spin-down is a complex process that is not yet fully
understood and currently no physical model can accurately reproduce all the
data available.
For this reason, even physically motivated gyrochronology models cannot always
be used to reliably extrapolate into unexplored parameter space.
Physical models, when calibrated to data can provide insight into the physics
of stars however, if accurate and precise {\it prediction} of stellar
properties is desired, empirical models can have advantages over physical
ones.
For example, the data may reveal complex trends that cannot be reproduced with
our current understanding of the physical processes involved but may be
captured by more flexible data-driven models.
In addition, it is relatively straightforward to build an element of
stochasticity into empirical models, \ie\ to allow for and incorporate
outliers or noisy trends.
This may be particularly important for stellar spin down, which does not
always seem to behave predictably.
% For example, synchronized binaries often appear to be rapidly rotating single
% stars which do not fall on the gyrochronology sequence and are perhaps one of
% the most common cause of outliers.
% One can never be certain that the rotation period of any given star truly
% reflects its age: there is a chance that it is an outlier.
% The probability that a star is an outlier in rotation-color-age space can be
% modeled and incorporated easily into an empirical gyrochronology model.
% In a deterministic model, a star with a rotation period that misrepresents its
% age will have a precise but extremely inaccurate inferred age.
% If however, some allowance for outliers or instrinsic scatter is built into
% the models, such a star would be recognized as an outlier and would have a
% very imprecise but accurate inferred age.
% The relation between rotation period and age for M dwarfs is noisy, and a
% deterministic model would not be appropriate for these stars: it would lead to
% inaccurately predicted ages with over-optimistic precision.
% Instead, these data could be appropriately modeled with a stochastic mixture
% of broad Gaussians, or similar.
A further advantage of empirical models is that inference is more tractable:
it can be extremely fast to fit them to data.
We use a simple, empirical, deterministic gyrochronology model in this work,
which, like any other gyrochronology model, cannot yet reproduce all the
observed data.
Simple modifications could be made to this model to produce
significant improvements, for example, by including instrinsic scatter and
outliers, however we leave these improvements for a future project.
% For example, an allowance for outliers; stars with anomalously fast or slow
% rotation periods, could be build into our model.
Ultimately, the model we present here will provide a baseline against which
other gyrochronology models can be compared.

% Despite significant advances in both theoretical and empirical gyrochronology
% models, no existing model is currently able to reproduce the data.
% % theoretical models of stellar spin-down as well as new calibrations of
% % empirical models, the gyrochronology relations have not yet been finalized
% % because
% New high quality rotation periods recently measured from Kepler/K2
% observations reveal that rotational evolution is more complicated than
% previously thought, partly because the shape of the rotation-color relation is
% different and unique for each cluster \citep{meibom2011, meibom2015,
% rebull2016, douglas2016, rebull2017, douglas2017} and partly because stars
% appear to stop spinning down at old ages \citep{angus2015, vansaders2016,
% vansaders2018}.
% Theoretical models have been developed to reproduce the lack of magnetic
% braking at old ages \citep{vansaders2016, vansaders2018}, however no model has
% yet been designed to fit the individual shape of every cluster.
% In addition, the relation between rotation period, color and age appears to be
% stochastic at all ages and masses, particularly for young and low-mass
% stars.
% The gyrochronology relations could be improved in three main ways.
% Firstly,
% because stellar evolution is a noisy process that produces outliers and
% scatter, gyrochronology models could be non-deterministic.
% Secondly, the gyrochronology models could be more flexible in order to
% capture the inter-cluster variation in the shape of the rotation-color
% relations.
% Finally, new calibrators are needed to more precisely constrain rotational
% evolution at low masses and old ages.
%  % for two main reasons.
% % Firstly, the rotational evolution of stars is complex and not well understood.
% % It is difficult to reproduce the trends in the data using the known physical
% % processes acting within stellar interiors, surfaces and winds.
% % It is also challenging to come up with an empirical model that is flexible
% % enough to capture trends in the data.
% % Secondly, there is a lack of suitable calibration stars with precisely
% % measured ages, particularly at old ages.  % and low masses.
% % % These regions of parameter space are especially important because some
% % % evidence suggests that rotational evolution goes through a transition at old
% % % ages and low masses.
% % % This region of parameter space is especially important because new evidence
% % % This region of parameter space is


% Paper outline
%-----------------------------------------------------------------------------
This paper is laid out as follows.
In section \ref{section:method} we describe our new age-dating model and its
implementation, in section \ref{section:results} we test this model on
simulated stars, cluster stars and asteroseismic stars, and in section
\ref{section:discussion} we discuss the implications of these tests and future
pathways for development.
Throughout this paper we use the term {\it `observables'} to refer to the
following observed properties of a star, \teff, \logg, observed bulk
metallicity ($[\hat{\mathrm{Fe/H}}]$), parallax ($\bar{\omega}$), photometric
colors in different passbands ($\bf{m_x}$) and rotation period
($P_{\mathrm{rot}}$).
The term {\it `parameters'} refers to the physical properties of that star:
age ($A$), mass ($M$), true bulk metallicity (\feh), distance ($D$) and V-band
extinction ($A_V$).
These are the properties that generate the observables.
