\section{Results}
\label{section:results}

% Intro
% TEST 1: simulations
%-----------------------------------------------------------------------------
%   - The simulated data
%   - The results
% PLOT: histograms over inferred ages (1) just isochrones (2) just
% gyrochronology, (3) both.

% TEST 2: Clusters
%-----------------------------------------------------------------------------
%   - The cluster data
%   - The results
% PLOT: histograms over inferred ages (1) just isochrones (2) just
% gyrochronology, (3) both.

% TEST 3
% Test the method on asteroseismic dwarfs.

% Rolling out/future
%   - Adding new datasets/dimensions/methods.
%-----------------------------------------------------------------------------

% Intro
%-----------------------------------------------------------------------------
In order to demonstrate the functionality of our method, we conduct a series
of tests.
In the first we simulate a set of observables from a set of fundamental
parameters for a few hundred stars using the MIST \citep{choi} stellar
evolution models.
In the second we test our model by attempting the measure the ages of stars in
known clusters who's ages have been established through ensemble isochrone
fitting and MS turn-off.
In the third test we compare our ages with those of asteroseismic measurements
for around twenty stars.

% TEST 1: simulations
%-----------------------------------------------------------------------------
%   - The simulated data
For the first test we began with a set of 1000 stars and drew masses, ages,
bulk metallicities, distances and extinctions at random from the following
uniform distributions:
\begin{eqnarray}
& M \sim U(0.5, 1.5)~[M_\odot] \\
& A \sim U(0.5, 14)\mathrm{~[Gyr]} \\
& F \sim U(-0.2, 0.2) \\
& D \sim U(10, 1000)~\mathrm{[pc]} \\
& A_V \sim U(0, 1).
\end{eqnarray}
\teff, \logg, \fhat, {\bf \mx}, \pmega\ and B-V were then generated using
these stellar parameters with the MIST stellar evolution models \citep{choi}
and rotation periods, $P$ were generated from the gyrochronology relation in
equation \ref{eqn:gyro} with age, $A$, and B-V.
% Uncertainties.
We then performed cuts on these simulated stars to remove evolved stars and
stars that are too hot.
The rotation periods of evolved stars, defined here to be those with \logg\ >
4.5 begin to increase as soon as they turn off the MS and their radii start to
enlarge and cannot be modeled with the gyrochronology relation of equation
\ref{eqn:gyro}.
In addition, hot stars (defined as 6250 K < \teff) cannot be modeled using
equation \ref{eqn:gyro} because their convective envelopes are extremely
shallow and their magnetic fields are weaker, leading to a lack of magnetic
braking.
The rotation periods of these stars do not increase substantially during their
time on the MS.
After performing these cuts, 649 \racomment{update} stars remain in the sample
of simulated stars.
We attempted to measure their true periods using our method outlined in
section \ref{sec:method}.
For all stars, our initial guesses for the parameters are $M = 1M_\odot$, $A =
1$ Gyr, $F = 0$, $D = 500$ pc and $A_V = 0.1$.

%   - The results
The top panel of figure \ref{fig:sims_results} shows the results of using a
{\it purely isochrone} model to estimate the posterior PDFs over the
stellar ages of simulated stars.
The rotation periods of stars have not been incorporated into this model,
these posterior PDFs were obtained by isochrone fitting only.
In some cases (shown as points with large errorbars) there is no constraint on
the stellar age: the age of the star is consistent with all ages from 0-14
Gyrs.
The reason for this is that the temperatures and luminosities of stars do not
change very much on the main sequence.
The isochrones are tightly spaced in the MS region of the HR-diagram and, as a
result, even precisely measured temperatures and luminosities often do not
yield precise ages.
In other cases there is a moderate constraint on the age.
% Accuracy
% precision

In the second panel of figure \ref{fig:sims_results} the results of simply
using a gyrochronology model are demonstrated.
These ages have been {\it inferred} using the likelihood of equation
\ref{eqn:gyro_likelihood}, rather than calculated by plugging numbers into the
gyrochronology relation of equation \ref{eqn:gyro}.
This means that uncertainties on the rotation periods are propagated through
to the posterior.
For this experiment, the B-V colors were pre-computed using the MIST
isochronal model grid, rather than being jointly inferred at the same time as
the ages.
Again, the true stellar ages are plotted on the $x$-axis and the median values
of the recovered PDFs over ages are plotted on the $y$-axis.
Here, unlike in the pure isochrone fitting case, the recovered ages are
precise.
That is because, in the gyrochronology model, age is the only free parameter
as opposed to the isochrone model demonstrated in the top panel which has five
free parameters: age, mass, metallicity, distance and extinction {\it and}
these parameters are highly correlated.
It is also because gyrochronology isochrones (or gyrochrones) are more widely
separated relative to the observational uncertainties than the isochrones used
above.
In addition, there is no intrinsic scatter built into this gyrochronology
model; it is deterministic.
This means that the pair of measurements: rotation period and B-V, returns a
single-valued age, rather than a probability distribution over ages.
Because we are testing the gyrochronology-only model on {\it simulated} data,
these results look precise {\it and} accurate, but that is misrepresentive.
Inaccuracies would arise if the gyrochronology were incorrect or not precisely
calibrated in all parts of parameter space which is in fact the case in
reality.
However, since we simulated these data from the same model we used to recover
the ages, the resulting ages inferred are accurate by construction.

The third panel of figure \ref{fig:sims_results} shows the perfect middle
ground between precise but inaccurate gyrochronology and accurate but
imprecise isochrone models.

% TEST 4: with and without spectroscopy
Although inferred ages are likely to be more precise if spectroscopic
parameters are available (\teff, \logg\ and \feh), it is still possible to
place constraints on stellar ages if only photometric colors are available,
especially if the star has a precise parallax measurement.
Figure \ref{fig:just_photometry} demonstrates the decrease in precision when
only photometric colors (J, H and K), parallaxes and rotation periods are
used as observables.
The top panel of figure \ref{fig:just_photometry} shows an isochrones-only
model and the bottom panel shows the results of using isochrones {\it and}
gyrochronology.
When spectroscopic parameters are not available, including age information
from a stellar rotation period becomes extremely important as it is far more
age-sensitive than photometric colors.
The trade-off, however, is that stellar ages will become gyrochronology
dominated, and it is even more important to use an accurate gyrochronology
model.

% TEST 2: Clusters
%-----------------------------------------------------------------------------
%   - The cluster data
In order to test our model on real stars with known ages, we selected a sample
of cluster stars with precisely measured ages from ensemble isochrone fitting
and main sequence turn off.
The ages of open clusters can be measured much more precisely than field
stars for two main reasons.
Firstly, the stars have the same age (to within a few million years), so the
age of a cluster can be inferred with an increased precision that is
proportional to the square root of the number stars, relative to a single star
case.
In addition, stars in the same cluster form (we assume) from the same
molecular cloud and therefore have the same metallicity.
Since cluster stars have the same metallicity and age, stars fall on the same
isochrone and the main sequence turn
off can be identified.
We compiled rotation periods and spectroscopic parameters for members of
the Pleiades which is 150 million years old, Praesepe, a 650 Myr cluster,
NGC 6811, 1.1 Myrs and NGC 6819, 2.5 Myrs.
Rotation periods from the Pleiades \citep{rebull2016} and Praesepe
\citep{douglas2016} were obtained from frequency analysis of \ktwo\ data, and
rotation periods for NGC 6811 \citep{meibom2013} and 6819 \citep{meibom2015}
stars were measured from \kepler\ prime light curves.
We crossmatched stars with measured rotation periods from these clusters with
the Gaia DR2 catalog.
% Cutting outliers and limiting color range.
In order to fit a new gyrochronology relation to these data we restricted the
sample of cluster stars to the color range, 0.56 $<$ \gcolor\ $<$ 3 in order
to remove early F and late M dwarfs whos' rotation periods do not fall on the
`gyrochronology main sequence'.
Although it {\it may} be possible to crudely predict the ages of these stars
(at least the M dwarfs) from their rotation periods, the age-rotation-color
relation for these stars is very different to the FGK star relations and is
not the focus of this paper.
The rotation periods of the cluster stars in the restricted color range are
plotted against their \Gaia\ colors in figure \ref{fig:clusters}.

% Out of a total of N Hyades stars with rotation periods \citep{radick1987,
% radick1995, hartman2011, delorme2011, douglas2016}, N of them have precise
% spectroscopic parameters \citet{brewer}.
% The rest have Gaia photometry ($G$, $G_BP$ and $G_RP$ bands), Kepler
% photometry ($Kp$) and Gaia parallaxes.
% We only selected stars with Gaia DR2 radial velocity (RV) measurements, and
% required that their RVs had to be consistent with the cluster RV of around 40
% kms$^{-1}$.
% Two stars with RVs $< 30$ kms$^{-1}$ were removed from the catalog.

% The Cluster figure
\begin{figure}
  \caption{
The rotation periods of stars in open clusters, and the Sun, plotted against
    their \Gaia\ colors (\gcolor) in logarithmic space.
We fit a polynomial gyrochronology model to these data in order to predict
    rotation periods from \gaia\ colors and ages.
The result of this fit is plotted on top of the data at the ages of
    the cluster stars and the Sun.
}
  \centering
    \includegraphics[width=1.1\textwidth]{clusters.pdf}
\label{fig:clusters}
\end{figure}

We then inferred the age of each star using our model.
We did not force the Hyades members to have the same age since the aim of this
experiment was to reveal the precision and accuracy of our method by
quantifying the level of scatter in our predicted ages and identifying regions
of parameter space where the ages deviate from the established age for the
Hyades.

%   - The results
The results are shown in figure \ref{fig:cluster_results} which follows the
same layout as figure \ref{fig:sims_results}.
The top panels shows the ages inferred using an isochrone-only model, the
middle shows a gyrochrone-only model and the bottom shows an isochrone plus
gyrochrone model.
Once again, the top panel demonstrates that using an isochrone model alone
produces imprecise ages.
The middle panel shows ages recovered using only a gyrochronal model and it
reveals inaccuracies in the gyrochronology relation used here.

The bottom panel, once again, demonstrates the power of using both isochronal
and gyrochronal models together to provide a balance of precision and
accuracy.

% TEST 3: Asteroseismology
%-----------------------------------------------------------------------------
%   - The asteroseismic data
%   - The results
