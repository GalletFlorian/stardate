% Intro
% TEST 1: simulations
%-----------------------------------------------------------------------------
%   - The simulated data
%   - The results
% PLOT: histograms over inferred ages (1) just isochrones (2) just
% gyrochronology, (3) both.

% TEST 2: Clusters
%-----------------------------------------------------------------------------
%   - The cluster data
%   - The results
% PLOT: histograms over inferred ages (1) just isochrones (2) just
% gyrochronology, (3) both.

% TEST 3
% Test the method on asteroseismic dwarfs.

% Rolling out/future
%   - Adding new datasets/dimensions/methods.
%-----------------------------------------------------------------------------

% Intro
%-----------------------------------------------------------------------------
In order to demonstrate the functionality of our method, we conduct a series
of tests.
In the first we simulate a set of observables from a set of fundamental
parameters for a few hundred stars using the MIST \citep{choi} stellar
evolution models.
In the second we test our model by attempting the measure the ages of stars in
known clusters who's ages have been established through ensemble isochrone
fitting and MS turn-off.
In the third test we compare our ages with those of asteroseismic measurements
for around twenty stars.

% TEST 1: simulations
%-----------------------------------------------------------------------------
%   - The simulated data
For the first test we began with a set of 1000 stars and drew masses, ages,
bulk metallicities, distances and extinctions at random from the following
uniform distributions:
\begin{eqnarray}
& M \sim U(0.5, 1.5)~[M_\odot] \\
& A \sim U(0.5, 14)\mathrm{~[Gyr]} \\
& F \sim U(-0.2, 0.2) \\
& D \sim U(10, 1000)~\mathrm{[pc]} \\
& A_V \sim U(0, 1).
\end{eqnarray}
\teff, \logg, \fhat, {\bf \mx}, \pmega\ and B-V were then generated using
these stellar parameters with the MIST stellar evolution models \citep{choi}
and rotation periods, $P$ were generated from the gyrochronology relation in
equation \ref{eqn:gyro} with age, $A$, and B-V.
We then performed cuts on these simulated stars to remove evolved stars and
stars that are too hot.
The rotation periods of evolved stars, defined here to be those with \logg\ >
4.5 begin to increase as soon as they turn off the MS and their radii start to
enlarge and cannot be modeled with the gyrochronology relation of equation
\ref{eqn:gyro}.
In addition, hot stars (defined as 6250 K < \teff) cannot be modeled using
equation \ref{eqn:gyro} because their convective envelopes are extremely
shallow and their magnetic fields are weaker, leading to a lack of magnetic
braking.
The rotation periods of these stars do not increase substantially during their
time on the MS.
After performing these cuts, 460 \racomment{update} stars remain in the sample
of simulated stars.
We attempt to measure their true periods using our method outlined in section
\ref{sec:method}.
For all stars, our initial guesses for the parameters are $M = 1M_\odot$, $A =
1$ Gyr, $F = 0$, $D = 500$ pc and $A_V = 0.1$.

%   - The results
Figure \ref{fig:sims_gyro_and_iso} shows the results of using our model to
estimate the posterior PDFs over stellar ages, using simulated stars.
The true stellar ages are plotted on the $x$-axis and the median values of the
recovered PDFs over ages are plotted on the $y$-axis.

% TEST 2: Clusters
%-----------------------------------------------------------------------------
%   - The cluster data
%   - The results

% TEST 3: Asteroseismology
%-----------------------------------------------------------------------------
%   - The asteroseismic data
%   - The results
