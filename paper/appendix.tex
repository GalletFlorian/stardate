\section{Appendix}
\label{section:appendix}

\subsection*{Priors}
\label{section:priors}

We used the default priors in the {\tt isochrones.py} {\it python} package.
The prior over age was,
\begin{equation}
p(A) = \frac{\log(10) 10^{A}}{10^{10.5} - 10^8}, ~~~ 8 < A < 10.5.
\end{equation}
% where $A$, is $\log_{10}(\frac{\mathrm{Age}}{\mathrm{yrs}})$.
where $A$, is $\log_{10}(\mathrm{Age~[yrs]})$.
% The prior over mass is uniform in natural-log between -20 and 20,
% \begin{equation}
%     p(M) = U(-20, 20)
% \end{equation}
% % where $M$ is $\ln(\frac{\mathrm{mass}}{M_\odot})$.
% where $M$ is $\ln(\mathrm{Mass}~[M_\odot])$.
The prior over EEP was uniform with an upper limit of 800.
We found that adding this upper limit reduced some multi-modality caused by
the giant branch and resulted in better performance.
The prior over true bulk metallicity was based on the galactic metallicity
distribution, as inferred using data from the Sloan Digital Sky Survey
\racomment{citation}.
% It is based on two double-Gaussian distribution, where the halo is described as
% a broad Gaussian and the galactic disc as a narrow Gaussian.
It is the product of a Gaussian that describes the metallicity distribution
over halo stars and two Gaussians that describe the metallicity distribution
in the thin and thick disks:
\begin{eqnarray}
    p(F) =
    & H_F \frac{1}{\sqrt{2\pi\sigma_{\mathrm{halo}}^2}}
    \exp\left(-\frac{(F-\mu_{\mathrm{halo}})^2}{2\sigma_{\mathrm{halo}}}\right)
    \\ \nonumber
    & \times (1-H_F)
    \frac{1}{\xi}
    \left[\frac{0.8}{0.15}\exp\left(-\frac{(F-0.016)^2}{2\times 0.15^2}\right)
    + \frac{0.2}{0.22}\exp\left(-\frac{(F-0.15)^2}{2\times
    0.22^2}\right)\right],
\end{eqnarray}
where $H_F = 0.001$ is the halo fraction, $\mu_\mathrm{halo}$ and
$\sigma_{\mathrm{halo}}$ are the mean and standard deviation of a Gaussian
that describes a probability distribution over metallicity in the halo, and
take values -1.5 and 0.4 respectively.
% $\mu_\mathrm{disk, 1}$, $\mu_\mathrm{disk, 2}$, $\sigma_\mathrm{disk, 1}$
% and $\sigma_\mathrm{disk, 2}$ are the means and standard deviations of two
The two Gaussians inside the square brackets describe probability
distributions over metallicity in the thin and thick disks.
The values of the means and standard deviations in these Gaussians are from
\citet{casagrande2011}.
$\xi$ is the integral of everything in the square brackets from $-\infty$ to
$\infty$ and takes the value $\sim 2.507$.
% D_F = 0.8 \sigma_{\mathrm{disk, 1}} = 0.15 \mu_{\mathrm{disk, 1}} = 0.016
% \sigma_{\mathrm{disk, 2}} = 0.22 \mu_{\mathrm{disk, 2}} = 0.15
The prior over distance was,
\begin{equation}
    p(D) = \frac{3}{3000^3} D^2, ~~~ 0 < D < 3000,
\end{equation}
with D in kiloparsecs, and, finally, the prior over extinction was uniform
between zero and one,
\begin{equation}
    p(A_V) = U(0, 1).
\end{equation}
