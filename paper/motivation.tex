\section{Combining isochrone fitting with Gyrochronology: Motivation}
\label{section:motivation}

In order to demonstrate why a combination of gyrochronology and isochrone
fitting can provide more precise ages than either method used in isolation, we
calculated the information provided by each method over the color-magnitude
diagram.
The decrease in rotation period with time is roughly proportional to the
inverse square root of age, $\frac{dP_{\mathrm{rot}}}{dt} \propto
\mathrm{Age}^{-n}$, where n$\sim$0.5 \citep{skumanich1972}.
This is a large rate of change relative to typical rotation period measurement
uncertainties.
For example, the Sun's rotation period is currently decreasing at a rate of
around 3 days per billion years \citep[unless it has already stopped spinning
down][]{vansaders2016}, and the 1 billion year-old Sun spun down at a rate of
around 6 days per billion years.
These are relatively large changes compared with the average uncertainties on
rotation period measurements: the median rotation period uncertainty in the
\citet{mcquillan2014} catalog is around 0.1 days.
In contrast, the temperature of a K dwarf changes by about 20 K every billion
years which is small compared to typical observational uncertainties of 20-100
K.
Rotational isochrones, or `gyrochrones' are more widely spaced and provide
much more {\it information} about age than traditional isochrones.
The difference in information conveyed by rotation vs \teff\ and $L$ can be
quantified by calculating the time derivatives of a star's observables.
The rate of change of \teff\ and $L$ dictates the minimum theoretical
uncertainty on an age calculated via isochrone fitting, given some
observational uncertainties.
Similarly, the rate of change of rotation period dictates the minimum
uncertainty on an age inferred via gyrochronology.
In order to quantify the minimum theoretical uncertainty on ages calculated
via isochrone fitting and gyrochronology, we calculated the Fisher information
for the MIST isochrones \citep{paxton2011, paxton2013, paxton2015, dotter2016,
choi2016, paxton2018} and an empirical polynomial gyrochronology model, fit to
the Praesepe cluster.
The Fisher information quantifies the amount of information that an observable
imparts onto an unknown parameter.
In the case of isochrone fitting, the observables are $G$ and \gcolor\ and the
parameter is age, or time, $t$.
The Fisher information is the variance of the parameter, $t$, given the
covariance of the observables and their derivatives with respect to $t$.
The inverse covariance matrix of the parameters (in this case we have just one
parameter, age or time, $t$), given the covariance matrix of the data, ${\bf y}
= [G, G_{BP} - G_{RP}$, is given by the following equation,
\begin{equation}
    C_{t}^{-1} = \left[\frac{d{\bf y}}{dt}\right]^T
    C_{\bf y}^{-1} \left[\frac{d{\bf y}}{dt}\right].
\end{equation}
Since we just have one parameter, $C_t^{-1}$ is a scalar, the inverse variance
of age, $\sigma_{\mathrm{Age}}^{-2}$.
In order to calculate the age uncertainty from the MIST isochrones and
gyrochronology model, we calculated numerical derivatives of $\frac{dG}{dt}$,
and $\frac{d(G_{BP} - G_{RP})}{dt}$ at every point on the MIST model grids.
We then calculated the age uncertainty, $\sigma_{\mathrm{Age,~iso}}$ at every
point on the grid.
Figure \ref{fig:iso_fisher} shows Solar-metallicity MIST isochrones, colored
by $\sigma_{\mathrm{Age,~iso}}$.

Figure \ref{fig:iso_fisher} shows the minimum theoretical absolute age
uncertainty, $\sigma_{\mathrm{Age,~iso}}$ (left panel), calculated using
typical \Gaia\ $M_G$ and \gcolor\ uncertainties represented as black errorbars
in the top right corner.
The typical \Gaia\ uncertainties are $0.5$ in both $M_G$ and \gcolor.
These estimates are based on a calculation of the median uncertainty on \Gaia\
absolute G-magnitude of cool stars which is dominated by the parallax
uncertainty.
We assumed the same uncertainty on \gcolor.
The minimum uncertainty on isochronal age ranges from around 10 million years
at MS turn off (upper left yellow area) to around the age of the Universe for
K dwarfs.
The minimum {\it relative} age uncertainty,
$\sigma_{\mathrm{Age,~iso}}/\mathrm{Age} \times 100$, plotted in the right
panel ranges from less than 1\% for old MS turn off stars which have ages
around 13 Gyr and age uncertainties less than 0.1 Gyr, up to tens-of-thousands
of percent for the youngest K and M dwarfs with unconstrained ages.

We also calculated the Fisher information for a combined isochronal and
gyrochronology model.
In this case we effectively had four observables: $G$ and \gcolor, determined
by the MIST isochrones; and $P_{\mathrm{rot}}$ (rotation period) and \gcolor\
{\it again}, this time determined by the gyrochronology model.
We used a simple gyrochronology model, calibrated by fitting a fourth-order
polynomial in rotation period-\Gaia\ color space and a first order polynomical
in rotation period-age space to the 625 Myr Praesepe cluster and the Sun,
only.
This model is described in more detail in section \ref{section:results}.
We calculated analytic derivatives for $\frac{dP_{\mathrm{rot}}}{dt}$ and
$\left(\frac{d(G_{BP} - G_{RP})}{dt}\right)_{\mathrm{gyro}}$ and combined
these with the numerical derivatives of $\frac{dG}{dt}$ and
$\left(\frac{d(G_{BP} - G_{RP})}{dt}\right)_{\mathrm{iso}}$ in order to
calculate the total age uncertainty, $\sigma_{\mathrm{Age,~(iso~\&~gyro)}}$.
Figures \ref{fig:iso_fisher} and \ref{fig:gyro_fisher} show Solar-metallicity
MIST isochrones, colored by $\sigma_{\mathrm{Age~(iso)}}$ and
$\sigma_{\mathrm{Age~(iso~\&~gyro)}}$.

Figure \ref{fig:gyro_fisher} shows the same results as figure
\ref{fig:iso_fisher} where this time the age uncertainties are calculated
using isochrones and a polynomial gyrochronology model.
Here the age uncertainties were calculated using typical \Gaia\ $M_G$ and
\gcolor\ uncertainties, represented as black errorbars in the top right
corner, and rotation period uncertainties of 1 day.
The minimum theoretical absolute age uncertainty,
$\sigma_{\mathrm{Age,~(iso~\&~gyro)}}$ (left panel), ranges from tens of
millions of years for stars at MS turn off to a few billion years (up to
around 3 Gyr for old G dwarfs).
The very precise ages at MS turn off are still provided by isochrone fitting
-- the incredible precision achievable with isochrone fitting at MS turn off
dominates over the precision provided by gyrochronology.
The gyrochronology model used to calculate the Fisher information is not
appropriate for stars turning off the MS as it does not account for a rapid
decrease in rotation period that may be caused by the stellar radius
increasing \citep[see][]{vansaders2013}.
However, it provides an upper limit on $\sigma_{\mathrm{Age,~gyro}}$ which is
is, in any case, dominated by $\sigma_{\mathrm{Age,~iso}}$.
The right-hand panel of figure \ref{fig:gyro_fisher} shows the relative age
uncertainty achievable with joint isochronal and gyrochronal age inference.
Relative age uncertainty,
$\mathrm{Age}/\sigma_{\mathrm{Age~(iso~\%~gyro)}}\times 100$ ranges from less
than 1\% at MS turn off, where isochrones are widely spaced, to a maximum of
around 30\% for young G dwarfs, where gyrochrones are most tightly spaced.
The minimum relative age uncertainty on GKM stars on the MS is typically
around 20\% because gyrochronology predicts precise ages for these kinds of
stars.
Gyrochronology contributes most of the precision on the MS because rotation
period information dominates over $M_G$ information.
An important caveat of this demonstration is that this is the minimum
theoretical precision given the {\it adopted} gyrochronology model and, since
the model used for this calculation does not include intrinsic scatter )which
is particularly large for young stars), these minimum age uncertainty
calculations are over-optimistic, especially for young stars.
Similarly, our model does not account for weakened magnetic braking at old
ages \citep{vansaders2016} so is also optimistic for old dwarfs.
Still, these figures provide an idea of gyrochronology's potential to deliver
much more precise ages than are achievable with isochrone fitting alone for
cool MS stars.

% Asteroseismology can provide precise ages of {\it both} red giant and MS stars
% but due to the greater quantity of observations suitable for {\it red giant}
% asteroseismology, precise red giant asteroseismic ages outnumber MS ages.
% % The frequencies of acoustic pulsations reveal the densities of stars, and,
% % as
% % those densities are age-dependent, can provide precise stellar age
% % measurements.
% % \kepler, a high-precision space-based photometric survey produces light curves
% % of stars that are sometimes precise enough to detect brightness fluctuations
% % caused by these acoustic pulsations.
% The typical periods of red giant acoustic pulsations are long (on the orders
% of hours to weeks) and can be detected using \kepler's long cadence mode of
% one observation per thirty minutes, \kepler's standard and most common
% observing mode.
% In addition, the amplitudes of red giant pulsations are typically very large,
% significantly greater than \kepler's photometric noise floor.
% MS stars, on the other hand, oscillate with periods of just a few minutes and
% the long cadence \kepler\ observations, taken once every half-hour, are
% far above the Nyquist limit and not capable of resolving these pulsations.
% Asteroseismic measurements of MS stars can only be made when observed in
% \kepler's short cadence mode of one observation every minute.
% % Around two thousand stars were observed in short cadence mode during \kepler's
% % original mission.
% However, since the amplitude of pulsation scales with stellar radius, the
% majority of stars with asteroseismic ages successfully measured using \kepler\
% short-cadence observations, of which there are currently around 500
% \citep{chaplin2014}, are subgiants.
% Only $\sim$30 of these are truly on the MS.
% This may change soon however: \kepler's short cadence
% light curves have recently been reprocessed and new, precise ages for the {\it
% all} stars observed in short cadence mode (around 2000 in the original
% \kepler\ mission) may be measured and made available soon.

% %   - Introduce gyrochronology
% % A star like the Sun will increase in luminosity by only around a factor of two
% % before turning off the MS.
% % In addition, the Sun's temperature will only increase by around 100 K during
% % its $\sim$8 billion year MS lifetime.
% The Sun's rotation period will vary by almost an order of magnitude over its
% MS lifetime due to magnetic braking.
% In contrast, luminosity and temperature are not sensitive age proxies for
% Sun-like stars, are often inferred indirectly, and can be difficult to infer
% precisely.
% In addition, the precision with which the luminosity and temperature of a star
% can be measured is highly sensitive to its distance and the amount of
% extincting dust along the line of sight.
% Stellar rotation periods are much more sensitive to age than luminosity or
% temperature and can be measured precisely, with little dependence on distance
% and none on extinction, directly from \kepler\ light curves.
% % Incorporating rotation period measurements into isochrone fitting methods
% % provides additional information that allows for much more precise age
% % inference.
% Gyrochronology, the dating method that uses stellar rotation periods, has the
% potential to provide MS star ages that are precise to around 20\%
% \citep{epstein2014}.
% Due to the abundance of rotation periods of MS stars already provided by
% \kepler/\ktwo\ and the many more expected from future photometric surveys,
% gyrochronology is one of the most readily available methods for inferring
% precise stellar ages and, as such, has gained interest over the last few
% years.

% % Inconsistency and inaccuracy of models
% In addition to the difficulties imposed by the slow timescale for change
% within MS stars that results in poor age precision, different dating methods
% often produce inconsistent predictions for the age of a star as a result of
% model inaccuracies.
% For example, an asteroseismic age will not necessarily agree with a isochronal
% one and even isochronal ages derived from different stellar evolution models
% can be inconsistent.
% % physics driving models are wrong.
% % Not enough calibration data.
% This problem arises from a lack of calibrators with sufficiently precise
% stellar properties.

% Bayesian isochrone fitting.
% This section should contain more background about the poor uncertainties
% expected when a purely isochrone fitting technique is used.

% Sum up.
%-----------------------------------------------------------------------------
% There is still some way to go before the models we use to age-date
% MS stars produce precise and accurate ages.
% This paper presents an incremental step toward improving age-dating models and
% best-practises.
% It also provides a discussion on the balance between accuracy and precision
% within age-dating models.

\begin{figure}
  \caption{
    This figure shows Solar-metallicity MIST isochrones in \Gaia\ absolute
    G-band magnitude and Gaia $G_{BP} - G_{RP}$ color.
    In the left panel the isochrones are colored by the minimum absolute age
    uncertainty at each point on the CMD, calculated using the Fisher
    information, based on the typical uncertainties of \Gaia\ photometry
    (represented by black errorbars in the upper right).
    The Sun's position \citep{casagrande2018} is indicated with the Solar\
    symbol.
    The purple color in the top left corresponds to small age uncertainties,
    \ie\ good age precision.
    Age precision increases as stars begin to turn off the MS.
    On the MS however, particularly at low masses, the age precision is poor.
    For late K dwarfs, for example, isochrone fitting age uncertainties exceed
    the age of the Universe.
    This makes sense when you consider that typical \Gaia\ uncertaintes on $G$
    and $G_{BP} - G_{RP}$ exceed the entire width of the MS, which spans
    0.01-14 Gyrs.
    Isochrone fitting is not an appropriate age-dating method for MS stars,
    especially at low masses.
    In the right panel the isochrones are colored by the logarithmic
    {\it relative} age precision at each point in the CMD.
    Relative age uncertainties range from 100\% for the oldest MS GKM stars,
    to several thousand percent for the youngest.
    These age uncertainties were calculated using the derivatives of $G$ and
    $G_{BP} - G_{RP}$ with age, \ie\ the rate of change in a star's luminosity
    and temperature.
    For example, K dwarf temperatures increase at a rate of only around 20 K
    per billion years.
    The precision with which an age can be measured is related to the
    separation between isochrones, which indicate epochs of rapid change
    (steep gradients).\label{fig:fischer_iso}
}
  \centering
    \includegraphics[width=1\textwidth]{iso_fisher_temp.png}
\label{fig:iso_fisher}
\end{figure}

\begin{figure}
  \caption{
    As figure \ref{fig:iso_fisher}, however in this case the minimum age
    uncertainties are calculated based on isochrone fitting and gyrochronology
    {\it combined}.
    The isochrones are colored by the minimum relative age uncertainty at each
    point on the CMD, based on the typical uncertainties of Gaia photometry
    and Kepler rotation period uncertainties (assumed to be around 1 day).
    In contrast to figure \ref{fig:iso_fisher}, here the left panel shows the
    {\it logarithmic} absolute age and the right panel shows the linear
    relative age.
    The isochrones still supply precise ages at the MS turn off (purple, upper
    left) however, gyrochronology supplies precise ages on the MS (15 - 25\%
    relative precision).
    Gyrochronology and isochrone fitting complement each other and when used
    together, all subgiants and MS stars can have ages more precise than 30\%.
    \label{fig:fischer_gyro}
}
  \centering
    \includegraphics[width=1\textwidth]{gyro_fisher_temp.png}
\label{fig:gyro_fisher}
\end{figure}
