\section{Discussion}
\label{section:discussion}

In the previous sections we demonstrated that modeling the ages of stars using
isochrones {\it and} gyrochronology can result in more precise ages than using
isochrone fitting alone.
Isochrone fitting and gyrochronology are  complementary because gyrochronology
is more precise where isochrone fitting is less precise (on the MS) and vice
versa (at MS turn off).
% Age precision is determined by the spacing of isochrones or gyrochrones: in
% regions where iso/gyrochrones are more tightly spaced, ages will be less
% precise.
% Isochrones becone less tightly spaced (and more precise) at larger stellar
% masses and lower surface gravities.
% Gyrochrones become more tightly spaced (and less precise) at larger stellar
% masses.
%   - How will this be used/what will it be useful for?
The method we present here is available as a {\it python} package called \sd\
which allows users to infer ages from their available apparent magnitudes,
parallaxes, rotation periods and spectroscopic propertes in just a few lines
of code.
This method is applicable to an extremely large number of stars: late F, GK
and early M stars with a rotation period and broad-band photometry.
This already includes tens-of-thousands of \kepler\ and \ktwo\ stars and could
include millions more from \tess, \lsst, \wfirst, \plato, \gaia, and others in
future.
Although this method is designed for combining isochrone fitting with
gyrochronology, \sd\ can still be used without rotation periods, in
which case it will predict an isochrone-only stellar age.
% \sd\ is therefore applicable to all stars covered by the MIST isochrones:
% masses from 0.1 to 300 M$_\odot$, ages ranging from 100,000 years to longer
% than the age of the Universe, and metallicities from -4 to 0.5.
However, it is {\it optimally applicable} to stars with rotation periods,
otherwise the result will be identical to ages measured with {\tt
isochrones.py}.
%   - Where is it not useful?
However, \sd\ will often predict inaccurate ages for stars younger than around
500 million years, where stars are more likely to be rapidly rotating
outliers, and close binaries whose interactions influence their rotation
period evolution.
% These ages may still be precise even though they are inaccurate.
% Building a mixture model into \sd\ would allow these outliers to be identified
% and this is one of the main improvements to \sd\ that we plan to make in
% future.
% Since many stars with measurable rotation periods do not have precise
% spectroscopic properties, it is not always possible to tell whether a star
% falls within these permissable ranges of masses, surface gravities and rossby
% numbers.
% In addition, any given star, even if it does meet the criteria for mass,
% age, binarity, etc, may still be a rotational outlier.
Rotational outliers are often seen in clusters \citep[see \eg][]{douglas2016,
rebull2016, douglas2017, rebull2017} and many of these fall above the main
sequence, indicating that they are binaries.
When a star's age is not accurately represented by its rotation period, its
isochronal age will be in tension with its gyrochronal one, however, given the
high information content of gyrochronology, the gyrochronal age will dominate
on the MS.
% Figure \ref{fig:bimodal} shows the posterior PDF for a star with a
% misrepresentative rotation period.
% This star is rotating more rapidly than its age and mass indicate it should,
% so the gyrochronal age of this star is under-predicted.
% Situations like this are likely to arise relatively often, partly because
% rotational spin-down is not a perfect process and some unknown physical
% processes can produce outliers, and partly because misclassified giants, hot
% stars, M dwarfs or very young or very old stars will not have rotation periods
% that relate to their ages in the same way.
In addition, measured rotation periods may not always be accurate and can, in
many cases, be a harmonic of the true rotation period.
A common rotation period measurement failure mode is to measure half the true
rotation period.
The best way to prevent an erroneous or outlying rotation period from
resulting in an erroneous age measurement is to {\it allow} for outlying
rotation periods using a mixture model.
We intend to build a mixture model into \sd\ in future.
% As shown in figure \ref{fig:praesepe}, the gyrochronology model used here
% \citep{angus2015} does not provide a good fit to all available data.
% In future we intend to calibrate a new gyrochronology model that fits all
% available cluster and asteroseismic data.
% For now however, we simply warn users of these caveats and suggest that ages
% calculated using \sd\ are treated with appropriate caution.

% %   - Caveats and gotchas -- e.g. isochrones aren't 100% accurate.
% Throughout this manuscript we have referred to the `accuracy' of the
% isochronal models.
% In reality though, stellar evolution models are not 100\% accurate and
% different stellar evolution models, \eg, MIST, Dartmouth, Yonsei-Yale, etc
% will predict slightly different ages.
% The disagreement between these models varies with position on the HR diagram,
% but in general, ages predicted using different stellar evolution models will
% vary by around 10\%.
% We use the MIST models in our code because they cover a broader range of ages,
% masses and metallicities than the Dartmouth models.

% %   - How including the rotation period improves precision of all parameters.
% Our focus so far has been on stellar age because this is the most difficult
% stellar parameter to measure.
% However, if the age precision is improved, then the mass, \feh, distance and
% extinction precision must also be improved, since these parameters are
% strongly correlated and co-dependent in the isochronal model.
% Figure \ref{fig:mass_improvement} shows the improvement in relative precision
% of mass measurements from our simulated star sample.
